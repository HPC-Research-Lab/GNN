\section{Background and Motivation}
To facilitate our discussion, we first give background on sampling-based GNN training and its relation to stochastic compositional optimization. 

\subsection{GNN Computation}
%Numerous GNN architectures have been proposed in recent years. 
%Most of them follow a similar computation pattern~\cite{zhou2018graph}. 
%Suppose we are given a graph $\mathcal{G}(\mathcal{V}, \mathcal{E})$ of $N = |\mathcal{V}|$ nodes and
%$|\mathcal{E}|$ edges as input, where each node is associated with a feature
%vector $x_i$. 
The computation at each layer of a GNN is conducted in two steps: {\tt aggregate} and {\tt update}. 
For each node $v$ in the graph, the  {\tt aggregate} function gathers data from its neighboring nodes and returns the aggregation result as 
\begin{equation}
    z_v = \mathrm{Agg}_v(h_{ne[v]}, x_v, x_{ne[v]}). 
\end{equation}
Here, $h_{ne[v]}$ is the intermediate features of $v$'s neighbors from the previous layer, $x_v$ is the input feature of $v$, and $x_{ne[v]}$ is the input features of $v$'s neighbors. 
The {\tt update} function uses the aggregated value to produce the intermediate features of $v$ as 
%\vspace{-.3em}
\begin{equation}
%\vspace{-.2em}
    h_v = \mathrm{Upd}_v(z_v, x_v). 
\end{equation}
By stacking the intermediate features and the input features of all nodes, the computation at layer $l$ can be written as 
\begin{equation}
\label{eq:agg_upd}
\begin{split}
Z^{(l)}=\mathrm{Agg}(H^{(l-1)}, X), \;\;\;\;\;\;
H^{(l)} = \mathrm{Upd}(Z^{(l)}, X, W^{(l)}).
\end{split}
\end{equation}
Here, $H^{(l-1)}=[h^{(l-1)}_1,\ldots, h^{(l)}_N]$ denotes the intermediate features of all nodes at layer $l$$-$$1$, $Z^{(l)}\in \mathbb{R}^{N\times d_l}$ is the aggregated features of all nodes at layer $l$, $X = [x_1,\ldots,x_N]$ is the input features of all nodes, and $ W^{(l)}$ is the learnable weights. 
As an example, Graph Convolutional Network (GCN)~\cite{kipf2017semi} has $\mathrm{Agg}(H^{(l-1)}, X)=PH^{(l-1)}$ where $P$ is the normalized Laplacian matrix of the graph, and $\mathrm{Upd}(Z^{(l)}, X,  W^{(l)})=\sigma(Z^{(l)} W^{(l)})$ where $\sigma$ is a non-linear activation function.   
Many other GNNs can be expressed in this form with different definitions of $\mathrm{Agg}$ and $\mathrm{Upd}$~\cite{zhou2018graph}. 





 

\subsection{Sampling-based GNN Training as Stochastic Compositional Optimization}

When the graph is large, the neigbhbor aggregation operation $\mathrm{Agg}$ incurs a large overhead, making the training of GNNs computationally challenging. 
Therefore, prior work has proposed to replace the $\mathrm{Agg}$ function with a sampled neighbor aggregation operation $\widetilde{\mathrm{Agg}}$. 
By sampling the neighboring nodes, an unbiased estimate of $Z^{(l)}$ is computed at each layer, i.e., 
\begin{equation}
  \widetilde{Z}^{(l)}=\widetilde{\mathrm{Agg}}(H^{(l-1)}, X)
\end{equation}
with $\mathbb{E}[\widetilde{Z}^{(l)}]=Z^{(l)}$. 
%Some recent works have identified the connection between sampling-based GNN training  and multi-level stochastic compositional optimization (SCO)~\cite{cong2020minimal}. 
%Since the convergence property of algorithms for multi-level SCO has been well studied~\cite{zhang2019multi, yang2019multilevel,  chen2020solving}, this connection will allow us to study the convergence of GNN training with different neighbor sampling methods. 
%By sampling the neighboring nodes, an unbiased estimate $\widetilde{\mathrm{Agg}}$ is computed for the aggregate function. 
If we define the computation at layer $l$ of the original GNN as a function 
\begin{flalign}
    f^{(l)}(Z^{(l-1)}, W^{(l-1)}, ..., W^{(T)})&=[Z^{(l)}, W^{(l)}, ..., W^{(T)}]\\ \nonumber 
    &=[\mathrm{Agg}(\mathrm{Upd}(Z^{(l-1)}, X, W^{(l-1)})), W^{(l)}, ..., W^{(T)}]],
\end{flalign}
the computation with sampled neighbor aggregation can be written as a stochastic function
\begin{flalign}
\label{eq:forward}
f^{(l)}_{\xi_l}(\widetilde{Z}^{(l-1)}, W^{(l-1)}, ..., W^{(T)})&=[\widetilde{Z}^{(l)}, W^{(l)}, ..., W^{(T)}]\\ \nonumber 
&=[\widetilde{\mathrm{Agg}}(\mathrm{Upd}(\widetilde{Z}^{(l-1)}, X, W^{(l-1)})), W^{(l)}, ..., W^{(T)}]]
\end{flalign}
where ${\xi_l}$ represents the sampled neighbors at layer $l$. 
Since $\widetilde{Z}^{(l)}$ is an unbiased estimate of $Z^{(l)}$, 
we have $\mathbb{E}[f^{(l)}_{\xi_l}]=f^{(l)}$, and 
the computation of a $T$-layer GNN can be written as 
\begin{equation}
\label{eq:loss}
    F(\theta) = \mathbb{E}_{\xi_{T+1}}\left[ f_{\xi_{T+1}}^{(T+1)}\left( \mathbb{E}_{\xi_{T}}\left[f_{\xi_{T}}^{(T)}\left(...{E}_{\xi_1}[f^{(1)}(\theta)]...\right)\right] \right) \right]
\end{equation}
where $\theta=[X, W^{(1)}, ..., W^{(T)}]$, $f^{(T+1)}$ is the loss function, and $f_{\xi_{T+1}}^{(T+1)}$ corresponds to the estimated loss with mini-batch sampling. 
Note that we put all the learnable weights in $\theta$ to formulate the computation as a stochastic compositional function. 
Our goal is to minimize $F(\theta)$, which is exactly a multi-level SCO problem.  



\subsection{Large Memory Consumption with A Naive Implementation}
%The main difficulty of SCO is due to the composition of the stachastic functions. 
%We are not able to an unbiased estimate of the gradient of $F(\theta)$.  
SCO has been well studied in the past few years, and many algorithms with gauranteed convergence have been proposed~\cite{zhang2019multi, yang2019multilevel,  chen2020solving, yang2019multilevel, balasubramanian2020stochastic, chen2020solving, lian2017finite, wang2017accelerating, ghadimi2020single}. 
It seems straightforward to adopt these SCO algorithms to achieve faster training of GNNs. 
However, these algorithms have large memory consumption when applied to GNN training and cannot run on GPUs for large graphs.  

To see the problem, let us consider the implementation of a single-timescale SCO algorithm~\cite{balasubramanian2020stochastic} for GNN training. 
Formally, the algorithm is written as 
\begin{equation}
  \label{eq:alg_step1}
             y^{(1)}_{k+1} = (1-\beta_{k})y^{(1)}_k + \beta_{k}  f^{(1)}_{\xi_{1,k}}(\theta_{k}),
 \end{equation}
\begin{equation}
         \label{eq:alg_step2}
      y^{(l)}_{k+1} = (1-\beta_{k})y^{(l)}_k + \beta_{k}  f^{(l)}_{\xi_{l,k}}(y^{(l-1)}_{k+1}), \;\;\;\; 2\le l \le T,
 \end{equation}
 \begin{equation}
  \label{eq:alg_step3}
g_{k+1} = (1-\beta_{k})g_{k} + \beta_{k} \nabla f_{\xi_{1,k}}^{(1)}(\theta_k) \nabla f_{\xi_{2,k}}^{(2)}(y^{(1)}_k)\ldots \nabla  f_{\xi_{T,k}}^{(T)}(y^{(T-1)}_k)\nabla f_{\xi_{T+1,k}}^{(T+1)}(y^{(T)}_k),
\end{equation}
 \begin{equation}
  \label{eq:alg_step4}
\theta_{k+1} = \theta_{k} - \alpha_k g_{k+1}.
\end{equation}

The key idea is to store an auxiliary variable $y^{(i)}$ to maintain the moving average of the value of each composite function. 
Since $f^{(l)}_{\xi_l}$ returns the exact values of $W^{(l)}, ..., W^{(T)}$, we only need to maintain a moving average of $\widetilde{Z}^{(l)}$ for each layer\footnote{A more rigorious justification for this implementation can be found in the Supplmentary~\ref{}.}. 
The computation is shown in Figure~\ref{fig:oom}. 
The moving average of the aggregated features is stored in $\bar{Z}^{(l)}$ with each row for one node. 
In each iteration, some nodes (rows) are sampled, and the estimated aggregation results $\widetilde{Z}^{(l)}$ are merged into $\bar{Z}^{(l)}$ based on Formula (\ref{eq:alg_step1}). 
For the nodes that are not sampled, we simply multiply the corresponding rows of  $\bar{Z}^{(l)}$ by $(1-\beta_k)$. 
Since the number of rows of $\bar{Z}^{(l)}$ is the number of nodes in the graph, $\bar{Z}^{(l)}$ takes a large chunk of memory when the graph is large. 
For example, for training a 3-layer GCN on a graph with two million nodes, suppose the hidden state dimension $d_l=512$ and a floating point has 4 bytes, $\bar{Z}$ takes $3\times 2\text{M} \times 512 \times 4=12$GB of memory. 
All of the existing SCO algorithms need to maintain this moving average, which impedes their application to large-scale GNN training. 








